\documentclass[12pt,a4paper]{article}
\usepackage[normalem]{ulem}
\usepackage{enumitem}
\usepackage{geometry}
\usepackage{amssymb}
\usepackage{hyperref}
\usepackage{minted}
\usepackage{biblatex} %Imports biblatex package
\addbibresource{refs.bib} %Import the bibliography file
\providecommand{\tightlist}{%
  \setlength{\itemsep}{0pt}\setlength{\parskip}{0pt}}

\title{Hacking Hashing in Competitive Programming}
\author{Masaki Takeuchi}

\date{June 3, 2024}

\begin{document}
\maketitle
\section{Introduction}\label{introduction}

During Competitive Programming contests, participants compete for the
maximum number of problems solved and earliest time solved.

One of the most popular Competitive Programming websites, Codeforces
features a ``hacking'' system where after you have submitted a solution
that passes the tests provided by the problem-setter, you have the
option to try to find a counter-test for a submission by another
participant.

Although hashing is a very powerful tool for competitive programming,
it's necessary to take precautions like introducing runtime randomness
into a program to make it non-deterministic and thus more robust against
hacking in a contest.

I researched the two most popular applications of hashing in Competitive
Programming in C++: Hashtables and Polynomial String Hashing.

\section{Defining Robustness}\label{defining-robustness}

Due to the nature of hashing, for all hashes with more possible states
than hash values, with an unlimited number of attempts and computing
power, a collision can theoretically be found such that
\texttt{M\ !=\ M\textquotesingle{}} and
\texttt{H(M)\ ==\ H(M\textquotesingle{})}.

The Codeforces platform typically gives a participant +100 points for a
successful hacking attempt against another competitor and -50 points for
unsuccessful tries. In addition, the hacking period is typically around
12 hours after the end of a contest. These parameters give more
reasonable bounds for the number of attempts and the computing power of
an adversary.

A robust hashing solution would be one that cannot be hacked in a
handful of attempts using an average computer within 12 hours with high
probability.

\section{Pseudo-Random Number Generators
  (PRG)}\label{pseudo-random-number-generators-prg}

In order to make hashing solutions robust against hacking attempts,
runtime randomness can be added using a pseudo-random number generator
with a random seed.

\subsection{\texorpdfstring{\texttt{rand()} :
        BAD}{rand() : BAD}}\label{rand-bad}

The C-standard library function rand is legacy code that is not suitable
for most applications today. The function returns a library-dependent
value between 0 and RAND\_MAX but the C-standard only guarantees that
\texttt{RAND\_MAX} is at least \texttt{32767} or \(2^{15} - 1\).\cite{randmax}

\subsection{Mersenne Twister : GOOD}\label{mersenne-twister-good}

32-bit and 64-bit versions of the Mersenne Twister random-number
generator are included in the C++ standard library as mt19937. The GCC
compiler also includes SIMD-oriented fast Mersenne Twister sfmt19937.

Pros: 
\begin{itemize}
    \item Can be initialized with a single line due to being in the C++
standard library
    \item Easy to generate distributions by using with std::uniform\_int\_distribution
\end{itemize}

Cons:
\begin{itemize}
    \item Uses an excessive 19937 bits of state and slow \cite{Sebastiano_2019}
\end{itemize}

\noindent In the context of competitive programming where implementation speed is
a large factor, std::mt19937 is still a very good PRG.

\noindent Example Usage:
\begin{verbatim}
mt19937_64 rng(chrono::steady_clock::now().time_since_epoch().count());
#define uid(a, b) uniform_int_distribution<>(a, b)(rng)
\end{verbatim}

\noindent The \texttt{uid} macro generates a value between \texttt{{[}a,\ b{]}}.

\newpage
\subsection{Xoshiro : GOOD}\label{xoshiro-good}

New PRGS such as Xoshiro (xor, shift, rotate) only utilize a handful of
operations such as add, xor, shift, and rotate which each only take a
single cpu instruction on modern hardware.

Pros: 
\begin{itemize}
    \item Very fast in comparison to std::mt19937
    \item Small state (256 bits for xoshiro256 64-bit generator)
\end{itemize}

Cons:
\begin{itemize}
    \item Not in C++ standard library, requires additional code
    \item  Requires additional code to implement fair mapping to a range compared to simply using std::uniform\_int\_distribution with Mersenne Twister \cite{Lemire_2019}
    \item \href{https://github.com/takeuchi-masaki/hacking-hashing/blob/main/hashmap/reverse_splitmix.cpp}{Easily reversible}
\end{itemize}

\href{https://github.com/takeuchi-masaki/Competitive-Programming-Templates/blob/main/xoshiro256.cpp}{Example Template Code}

It is not worth using xoshiro or xoroshiro PRGS in competitive
programming even if template code is set up beforehand since the speed
difference is not significant for algorithm contests. 

However in Heuristics contests such as 
\href{https://atcoder.jp/?lang=en}{Atcoder Heuristic Contest}, the speed 
difference of the PRG allows for more iterations of non-deterministic 
algorithms like Simulated Annealing and thus can lead to a 
competitive advantage.

\section{Seeding PRGs}\label{seeding-prgs}

\subsection{Fixed seed : BAD}\label{fixed-seed-bad}

With a fixed seed, the sequence generated by a PRG is also fixed. The
sequence can be generated once to find an adversarial testcase.

\subsection{\texorpdfstring{\texttt{time(nullptr)} :
        BAD}{time(nullptr) : BAD}}\label{timenullptr-bad}

Like rand, time is part of the C standard library.

``The value returned generally represents the number of seconds since
00:00 hours, Jan 1, 1970 UTC (i.e., the current~\emph{unix timestamp})'' \cite{ctime}

The return value is a 32-bit value that increments every second. This
value is very predictable. By finding the fixed sequence for a time
range, for example it is possible to generate 60 collision sequences for
a time range of a minute and combine the results to generate an
adversarial testcase. As long as the test is ran on the server within
the next minute, this will result in a successful hacking attempt.

\subsection{\texorpdfstring{\texttt{std::chrono::steady\_clock} :
        GOOD}{std::chrono::steady\_clock : GOOD}}\label{stdchronosteady_clock-good}

\begin{verbatim}
chrono::steady_clock::now().time_since_epoch().count()
\end{verbatim}

This returns a 64-bit value with significantly higher precision in
comparison to the previous ctime. Although the top bits of the number
are very predictable like ctime, the lower bits are very unpredictable.
Even calling the function above twice in two successive lines in a local
program yields differing lower bits.

Since the lower bits are very volatile, combined with the avalanche
effect in good PRGs, it is not practical to take advantage of this value
when used as a seed.

\section{Hashtables}\label{hashtables}

\subsection{\texorpdfstring{How \texttt{std::unordered\_map}
        works}{How std::unordered\_map works}}\label{how-stdunordered_map-works}

The C++ standard library hashtable std::unordered\_map works in the
following way:\cite{Forisek_2014}
\begin{enumerate}
    \item std::hash is called on the key value.
    \item The hashed value is distributed to one of the buckets by taking hash \%
bucket-count. The bucket size is always a prime number.
    \item When different key values are mapped to the same bucket, std::unordered\_map uses separate chaining to deal with collisions.
    \item When the number of items in the data structure go over a set load factor, the bucket-count is increased.
\end{enumerate}

\subsection{Hacking}\label{hacking}

Several parts of this implementation make it vulnerable to hacking.\\
First, std::hash when called on integer values acts as the
\textbf{identity function}, meaning the value is not actually hashed at
all.\\
Second, the prime numbers used for the bucket-counts are fixed for each
compiler version.\\
Third, the load factor by default is 1.0, meaning that the hashtable
bucket-size does not increase until the number of elements equals the
previous bucket-size.

By calling std::unordered\_set::bucket\_count() or
std::unordered\_map::bucket\_count() while adding elements, it is
possible to find the set bucket sizes for the specific compiler version.

\begin{verbatim}
int find_bucket_count(int N) {
    unordered_set<int64_t> st;
    int sz = -1;
    for (int i = 0; i < N; i++) {
        st.insert(i);
        if (st.bucket_count() != sz) {
            sz = st.bucket_count();
            cout << st.size() << ": " << sz << "\n";
        }
    }
    return sz;
}
\end{verbatim}

Suppose a problem involves finding the number of duplicated values from
a sequence of \texttt{n} integers. With few collisions, the expected
running time using a hashtable is O(1) for each insertion and lookup
operation to process \texttt{n} numbers.

\begin{verbatim}
void run(int N, int bucket_size) {
    unordered_map<int64_t, int64_t> mp;
    int64_t curr = 0;
    for (int i = 0; i < N; i++) {
        mp[curr]++;
        curr += bucket_size;
    }
    int64_t sum = 0;
    for (int i = 0; i < N; i++) {
        curr -= bucket_size;
        sum += mp[curr];
    }
    cout << sum << "\n";
}
\end{verbatim}

\href{https://github.com/takeuchi-masaki/hacking-hashing/blob/main/hashmap/check_bucketsize.cpp}{Full example here}

By instead adding in equivalent values modulo the last bucket-count, an
adversary can force a collision chain of length \texttt{n}. Instead of
each add and lookup operations taking about \texttt{O(1)} time, the time
complexity could increase to up to \texttt{O(n)} for each operation and
cause a ``Time limit exceeded'' verdict.

Using \(n = 10^5\), the above code runs in around 10 seconds on a M1
Macbook, while a random set of numbers runs in well under a second.

\subsection{Robust solution}\label{robust-solution}

One way to make Hashtables robust against the previous attack is to
actually hash the input value. First created for the Java language, the
Splitmix PRG takes in a state value and applies xorshift operations to
create the next state.\cite{Steele_Lea_Flood_2014}

This can be added as a hash function to use with std::unordered\_set or
std::unordered\_map.

However, since
\href{https://github.com/takeuchi-masaki/hacking-hashing/blob/main/hashmap/reverse_splitmix.cpp}{Splitmix can easily be reversed} , runtime randomness must be additionally added
to the funciton. Without runtime randomness, instead of adding multiples
of the bucketcount, an adversary could add the inverse splitmix of the
multiples of the bucketcount.

\begin{verbatim}
struct custom_hash {
    constexpr static uint64_t splitmix64(uint64_t x) {
        x += 0x9e3779b97f4a7c15;
        x = (x ^ (x >> 30)) * 0xbf58476d1ce4e5b9;
        x = (x ^ (x >> 27)) * 0x94d049bb133111eb;
        return x ^ (x >> 31);
    }
    size_t operator()(uint64_t x) const {
        static const uint64_t FIXED_RANDOM = chrono::steady_clock::now().time_since_epoch().count();
        return splitmix64(x + FIXED_RANDOM);
    }
};
template<class K> using hashset = unordered_set<K, custom_hash>;
template<class K, class V> using hashmap = unordered_map<K, V, custom_hash>;
\end{verbatim}

This template code allows the usage of ``hashset'' in an identical
way to std::unordered\_set \cite{Wu_2018}

\section{Polynomial String Hashing}\label{polynomial-string-hashing}

Suppose you want to hash a string \texttt{s} of length \texttt{n} with
base \texttt{b}.

Let \(hash(s) = s[0] \cdot b^0 + s[1] \cdot b^1 + s[2] \cdot b^2 + ... + s[n - 1] \cdot b^{n - 1}\)

Polynomial string hashing uses the information of what character is at
what position in order to create a hash value. Because \(b^{n - 1}\)
quickly becomes larger than what would fit in an integer, all the
calculations are typically done modulo a large prime number.

Hashing allows for comparisons of long substrings by doing integer
comparisons in \texttt{O(1)} time. In addition, the Rabin-Karp Rolling
Hash technique makes it possible to modify a hash from one substring to
a neighboring substring in \texttt{O(1)} time, to make it even more
useful.

Although polynomial string hashing has a lot of applications to
competitive programming, it is also very susceptible to collisions.

\subsection{Birthday Attack}\label{birthday-attack}

Suppose many random numbers are picked within the range of \texttt{[1, N]}.
The birthday phenomenon describes that when the number of values chosen
becomes much larger than \(\sqrt{N}\), it is likely to see at least
one repeated value.

When the polynomial base and modulo are fixed, an adversary can generate
random strings and expect a repeated hash value after about
\(\sqrt{\texttt{mod}}\) strings tested since the hash will result in a value within \([0, \texttt{mod} - 1]\).

Note: In the test code, the string hashing is reversed as\\
\(s[0] \cdot b^{n - 1} + s[1] \cdot b^{n - 2} ... + s[n - 1]\) for ease
of implementation.

\subsection{Thue-Morse Sequence}\label{thue-morse-sequence}

The Thue-Morse sequence is ``a binary sequence obtained by starting with
0 and successivel appending the Boolean complement of the sequence so
far''.\cite{Wikipedia_2024}

This can be used to attack hashing solutions that let the result of
polynomial hashing overflow using the unsigned integer bound. 

Let the `0's of a Thue-Morse sequence of length of a power of 2 be replaced 
by 'A' and `1's with 'B'. This AB Thue-Morse string and its inverse (all 'A's
swapped with 'B's and vice versa) hash to the same value when the base
is odd and the length of the string becomes long enough.

\noindent Generating AB Thue-Morse strings \texttt{s}, \texttt{t} of length
\(2^q\).
\href{https://github.com/takeuchi-masaki/hacking-hashing/blob/main/string-collision/thue_morse_attack.cpp}{Full
    Example Code}

\begin{verbatim}
int n = 1 << q;
string s, t;
char ch[2] = { 'a', 'b' };
for (int i = 0; i < n; i++) {
    s.push_back(ch[__builtin_parity(i)]);
    t.push_back(ch[!__builtin_parity(i)]);
}
\end{verbatim}

\subsection{Other attacks against Fixed-Base and Fixed-Modulo
    hashes}\label{other-attacks-against-fixed-base-fixed-modulo-hashes}

Since the birthday attack takes about \(2^{30}\) operations against a
\(2^{60}\) modulus, hashing the same string twice with two different bases
or modulus, or double-hashing, can be an effective strategy against it. 
However, other attacks are able to quickly find collisions against 
64-bit modulos and double-hashing.

A codeforces user recently created a
\href{https://codeforces.com/blog/entry/129538}{website that runs a
    Lattice-Reduction attack} locally within the browser. For example, this
attack was able to generate a collision against a double hash with
modulo \(2^{61} - 1\) within 2 seconds, whereas it would be impractical
to attempt to find a collision using a birthday attack.

\subsection{Robust solution}\label{robust-solution-1}
Randomizing the base at runtime prevents offline hash collision attacks, as long
as the modulus is not a power of two (in which case the Thue-Morse Sequence can be utilized).

Randomizing the base for the polynomial string hashing at runtime and using a 64-bit modulo is enough to make polynomial string hashing robust against hacking.

\href{https://github.com/takeuchi-masaki/hacking-hashing/blob/main/string-collision/saferabinkarp.cpp}{Template code}

\printbibliography

\end{document}